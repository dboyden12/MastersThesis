%\documentclass[11pt, oneside]{report}   	% use "amsart" instead of "article" for AMSLaTeX format
\documentclass[./Thesis]{subfiles}
%\usepackage{geometry}                		% See geometry.pdf to learn the layout options. There are lots.
%\geometry{letterpaper}                   		% ... or a4paper or a5paper or ... 
%\geometry{landscape}                		% Activate for rotated page geometry
%\usepackage[parfill]{parskip}    		% Activate to begin paragraphs with an empty line rather than an indent
%\usepackage{graphicx}				% Use pdf, png, jpg, or eps§ with pdflatex; use eps in DVI mode
%\usepackage{subfiles}							% TeX will automatically convert eps --> pdf in pdflatex		
%\usepackage{amssymb}

%SetFonts

%SetFonts



\begin{document}
%\label{ch:intro}

\chapter{Introduction to the Muon G-2 Experiment}

The muon g-2 experiment has a long history of the associated science associated with it. In the following Sections I will explain the history of the experiment and where this experiment comes from along with the fundamental idea of this experiment.  Furthermore, I will explain the results that were done in the previous experiment E821 and how this relates to standard model predictions which give the fundamental reasoning for performing E989 the current G-2 experiment.

\section{History and Motivation}
In the following section I will give a brief explanation of the magnetic and dipole moment's and the motivation of the muon g-2 experiment.
\subsection{Magnetic and Electric Dipole Moments}

	The study of magnetic moments started with the development of quantum mechanics. For fermions it is related to the spin by 
	
	\begin{equation}\label{EQ:MDM}
	\vec{\mu} = g \frac{Qe}{2m} \vec{s}
	\end{equation}
	
In our modern interpretation of the Stern-Gerlach experiment is that their observations was telling us that the g-factor of the unpaired electron in 2.  However, coming to this conclusion required the discovery of the spin, quantum mechanics, along w/ Thomas' relativistic correction. Following this Dirac's relativistic theory predicts that $g=2$. In 1933 Stern showed that that the g-factor of the proton was approximately 5.5 which proved that the proton was not a pure Dirac particle. In addition, Alvarez and Bloch discovered the the neutron had a large magnetic moment which was not expected.

In 1947, motivated by measurements of the hyperfine structure in hydrogen, which obtained the splitting are larger than expected from Dirac theory, Schwinger showed that from a theoretical standpoint this splitting can be accounted for by adding in a term for the electron spin magnetic moment for the lowest radiative correction to the Dirac moment.

	\begin{equation}
	\frac{\delta\mu}{\mu}  = \frac{1}{2 \pi } \frac{e^2}{\hbar c}
	\end{equation}
	
It has been found useful to break the magnetic dipole moment into two terms 
	
	\begin{equation}
	\mu = (1+a)\frac{e\hbar}{2m}
	\end{equation}
	Where
	\begin{equation}
	a = \frac{g-2}{2}
	\end{equation}
	
	
\subsection{The Muon}
	The muon was first observed in a Wilson cloud chamber by Kunze in 1933. In 1936, Anderson and Neddermeyer reported that the muon is less massive than the proton but more penetrating than the electron. The Yukawa theory of the nuclear force predicted such a particle which interacted to weakly with matter to be the carrier of the nuclear force. Muon decays through the nuclear force $e^+ \rightarrow e^{-}\nu_{\mu}\overline{\nu}_{e}$ in which the muons long lifetime of approximately 2.2 ns permits precession measurements of its mass, lifetime, and magnetic moment which makes it a good candidate for experiments.

\subsection{The Muon Magnetic Moment}
	The muon magnetic moment played an important role in the discover of the generation structure of the standard model. The muon spin experiment at Nevis Cyclotron observed parity violation in muon decays, in addition to showing that $g_\mu$ is consistent with 2. Subsequent experiments showed that $a_\mu \equiv \frac{\alpha}{2\pi} $ implying that in a magnetic field the muon behaves like a heavy electron.


\subsection{The Muon Electric Dipole Moment}


	Dirac discovered an EDM term in his relativistic theory and like the magnetic dipole moment, the EDM must be in the direction of the spin. The equation for the Electric dipole moment is given by Eq. \ref{EQ:EDM}
	
	\begin{equation}\label{EQ:EDM}
		\vec{d} = \eta (\frac{Qe}{2mc}) \vec{s}
	\end{equation}

	Where $\eta$ is dimensionless and analogous to g in Eq. \ref{EQ:MDM}. An EDM is forbidden by parity and by time reversal which was first pointed out by Landau and Ramsey by examining the hamiltonian \ref{EQ:Hamiltonian}
	
	\begin{equation}\label{EQ:Hamiltonian}
	H = -\vec{\mu} \cdot \vec{B} - \vec{d} \cdot \vec{E}
	\end{equation}
	
Therefeore searches for a permanent EDM of electrons, neutrons, and atomic nucleus have become an important search for physics beyond the standard model.


\subsection{Quick Summary to Experimental Technique}

	A polarized beam of muon are produced and injected into a storage ring. The Magnetic field is a dipole field in the storage ring. Vertical focussing is provided by the electrostatic quadrapoles. Two frequencies are measured experimentally. The first frequency $\omega_{a}$ is the rate at which the muon polarization turns relative to the momentum. The second frequency $\omega_p$ is the value of the magnetic field normalized to the larmour frequency.
	
	\begin{equation}
	\vec{\omega}_{a} = \vec{\omega}_{S} - \vec{\omega}_{C}
	\end{equation}

Where S denotes the spin and C denotes the cyclotron which the individual terms are given by Eq.'s \ref{EQ:omegaS} and \ref{EQ:omegaC}

	\begin{equation}\label{EQ:omegaS}
	\omega_{S} = -g \frac{Qe}{2m} B - (1-\gamma)\frac{Qe}{\gamma m} B
	\end{equation}

	\begin{equation}\label{EQ:omegaC}
	\omega_C = - \frac{Qe}{m\gamma} B
	\end{equation}

It is worth noting that $\omega_a$ depends only on the anomoly $a_\mu$ and depends linearly on the applied magnetic feild.

In the presence of an electric field we get \ref{EQ:omegaA}

\begin{equation}\label{EQ:omegaA}
\vec{\omega}_a = -\frac{Qe}{m}[a_{\mu} \vec{B} + (a_{\mu} - (\frac{m}{p})^2) \frac{\vec{\beta} \times \vec{E}}{c}]
\end{equation}


If $p_{magic} = \frac{m}{\sqrt{a_\mu}} \equiv 3.09 \frac{Gev}{c}$ the electric feild contribution in Eq. \ref{EQ:omegaA} cancels to the 1st order which only requires higher order corrections to the term. The factor $a_\mu$ from the 2 frequencies we use the relation \ref{EQ:amu}

\begin{equation}\label{EQ:amu}
a_\mu = \frac{\omega_a / \omega_p}{\lambda_{+} - \omega_a / \omega_p}
\end{equation}

The necessary steps in the experiment consist of the following.


1.	Production of an appropriate pulsed proton beam from accelerator complex.

2.	Production of pions using the proton beam that has been prepared.

3. 	Collection of the polarized muons from pion decay.

4.	Transporting GM2 storage ring.

5. 	Injection of the muon beam into the storage ring.

6. 	Kicking the muon beam onto stored orbits

7.	Measuring the arrival time and energy of positrons form the muon decays.


\vspace{5mm}

E989 Must obtain 21 times more data than the E821 experiment. Using the T method to evaluate the uncertainty $1-8 * 10^11$ events are required in the final fitted histogram to realize a 0.1ppm statistical uncertainty. $\omega_{a}$ and $\omega_{p}$ must be reduced down to the 0.07ppm level. The Statistical errors in the experiment come from the least squares fits to the histogram describing decay electron events vs. time and the amount of Data accumulation. The systematical errors in the experiment arise from anything that might cause the extracted frequency value from the true fit value which may consist of something as gain instability in the detectors.  In addition, the incoming beam line consists of multiple different systematical errors such as lost muons, spin tracking, coherent betatron oscillations, differential decays, pitch correction uncertainties. The main error that consists of the $\omega_{p}$ measurement is the level at which magnetic fields are able to be determined. 

\section{Standard Model Predictions}




\section{Results from E821}

	The muon g-2 experiment that took place at BrookHaven national laboratory resulted in a measurement variance from the standard model within 3 sigma measurement. The value for $a_{\mu}$ 




\section{Methodology of Measurement and requirements}




\end{document}